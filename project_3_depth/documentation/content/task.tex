\chapter{Task}
For the third and final compulsory project of the lecture Computer Vision \cite{cv_lecture_ex}, the dimensions and distance of objects shot by a camera should be estimated. In the Moodle course, two videos are provided, with each containing a chessboard to do the camera calibration. Video 1 also contains a Schwalbe motor roller; video 2 a 3D printer. Additionally, a test video (video 0) is provided which contains the chessboard in known distance of 250 cm to the camera. This video can be used to verify the results of the calculation process.

Using the intrinsic camera matrix and the known size of the chessboard (50x50 cm), the sizes and distances of the real-world objects should be calculated.

Necessary goals of the task as stated in the provided exercise document are: \cite{cv_lecture_ex}
\begin{itemize}
    \item Provide the geometrical/mathematical reasoning how you did calculate the distance, width, and height.
    \item For the distance estimation, follow the approach above by using the two 
    lines calculated with the help of the intrinsic matrix. Provide sketches where needed.
    \item Calibrate the camera for each video separately.
    \item Provide your distance estimation for the Schwalbe and the 3D metal printer.
    \item Provide your width and height estimation for the 3D metal printer.
\end{itemize}

The theoretical background is covered in the lecture and will not be explained in detail in this project documentation.
